\documentclass{article}
\newcommand{\blattNr}{7}
\usepackage[utf8]{inputenc}
\usepackage[ngerman]{babel}
\usepackage{enumitem}
\usepackage{listings}
\usepackage{lstautogobble}
\usepackage{amsmath}
\usepackage{xcolor}

\definecolor{mygreen}{rgb}{0,0.6,0}
\definecolor{keyword}{HTML}{CB772F}
\definecolor{string}{HTML}{A5C25C}

\lstset{literate=
    {Ö}{{\"O}}1
    {Ä}{{\"A}}1
    {Ü}{{\"U}}1
    {ß}{{\ss}}1
    {ü}{{\"u}}1
    {ä}{{\"a}}1
    {ö}{{\"o}}1
    {~}{{\textasciitilde}}1,
    basicstyle=\small\ttfamily,
    numbers=left,
    breaklines=true,
    commentstyle=\color{mygreen},
    frame=single,
    keywordstyle=\color{keyword},
    stringstyle=\color{string}, 
}
\newcommand{\java}{\lstinline[language=Java]}

\setcounter{section}{\blattNr}

\newcommand{\uebungsblattTitel}{
    \title{EidI Übungsblatt \blattNr}
    \author{Jonas Otto, Marco Deuscher}
    \maketitle
}

\begin{document}
    \uebungsblattTitel
    \paragraph{Methode1}
    \ \\
    \begin{lstlisting}[language=Java, autogobble]
    static int methode1(int[] arr){
        int min = 100;
        int minidx = 0;
        for (int i = 0; i < arr.length; i++) {
            if(arr[i]< min ){
            minidx = i;
            }
        }
        return minidx ;
    }
    \end{lstlisting}

    \begin{tabular}{r|l}
        Zeile & Schritte \\
        \hline
        2 & 1 \\
        3 & 1 \\
        4 & 1, arr.length + 1, arr.length \\
        5 & arr.length \\
        6 & Worst case: arr.length \\
        \hline
        Summe & $T(\text{arr})=4+4*\text{arr.length} = O(\text{arr.length})$
    \end{tabular}

    \paragraph{Methode2}
    \ \\
    \begin{lstlisting}[language=Java, autogobble]
    static int methode2(int n) {
        int count = 2;
        for (int i = 1; i <= n; i++) {
            for (int j= n; j > i; j--) {
                count++;
            }
        }
        return count;
    }
    \end{lstlisting}

    \begin{tabular}{r|l}
        Zeile & Schritte \\
        \hline
        2 & 1 \\
        3 & 1, $n + 1$, $n$ \\
        4 & $n$, $n$ \\
        5 & $n^2 + 1$ (1)\\
        \hline
        Summe & $T(n)=n^2+4n+4 = O(n^2)$
    \end{tabular}

    \vspace{1cm}
    In for schleife:
    \begin{eqnarray}
        n + 2 \cdot \sum_{i=1}^nn-i \enspace + 1
        &=& n + 2 \cdot (\sum_{i=1}^nn - \sum_{i=1}^ni) + 1 \nonumber\\
        &=& n + 2n^2 - 2 \cdot (\frac{n^2}{2} + \frac{n}{2}) +1 \nonumber\\
        &=& n^2 +1
    \end{eqnarray}

\end{document}
