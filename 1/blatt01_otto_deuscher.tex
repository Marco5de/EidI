\documentclass{article}

\usepackage{amsmath}
\usepackage{enumitem}

\usepackage{listings}
\lstset{literate=
    {Ö}{{\"O}}1
    {Ä}{{\"A}}1
    {Ü}{{\"U}}1
    {ß}{{\ss}}1
    {ü}{{\"u}}1
    {ä}{{\"a}}1
    {ö}{{\"o}}1
    {~}{{\textasciitilde}}1
}

\usepackage[utf8]{inputenc}

\begin{document}
  \title{EidI Übungsblatt 1}
  \author{Jonas Otto, Marco Deuscher}

  \maketitle

  \section{}
    \subsection{Aufgabe 1.1}
      \begin{itemize}
        \item $101010_{(2)}$\\
          Polynom:
          \begin{equation}
            1 \cdot 2^5 + 0 \cdot 2^4 + 1 \cdot 2^3 + 0 \cdot 2^2 + 1 \cdot 2^1 + 0 \cdot 2^0
          \end{equation}
          Horner Schema:
          \begin{equation}
            ((((1 \cdot 2 + 0) \cdot 2 + 1) \cdot 2 + 0) \cdot 2 + 1 ) \cdot 2 + 0 = 42
          \end{equation}
        \item $157_{(8)}$\\
          Polynom:
          \begin{equation}
            1 \cdot 8^2 + 5 \cdot 8^1 + 7 \cdot 8^0
          \end{equation}
          Horner Schema:
          \begin{equation}
            (1 \cdot 8 + 5) \cdot 8 + 7 = 111
          \end{equation}
        \item $ACDC_{(16)}$\\
          Polynom:
          \begin{equation}
            10 \cdot 16^3 + 12 \cdot 16^2 + 13 \cdot 16^1 + 12 \cdot 16^0
          \end{equation}
          Horner Schema:
          \begin{equation}
            \begin{aligned}
              ((A_{(16)} \cdot 16 + C_{(16)}) \cdot 16 + D_{(16)}) \cdot 16 + C_{(16)} &=\\
              ((10 \cdot 16 + 12) \cdot 16 + 13) \cdot 16 + 12 &= 44252
            \end{aligned}
          \end{equation}
        \item $10_{(10)}$\\
        Polynom:
        \begin{equation}
          10_{(10)} = 1 \cdot 2^3 + 0 \cdot 2^2 + 1 \cdot 2^1 + 0 \cdot 2^0
        \end{equation}
        Horner Schema:
          \begin{equation}
            \begin{aligned}
              (1 \cdot 10) + 0 &=\\
              (1_{(2)} \cdot 1010_{(2)}) + 0 &= 1010_{(2)}
            \end{aligned}
          \end{equation}
      \end{itemize}
    \subsection{Aufgabe 1.2}
      \paragraph{Problemspezifikation} Aus einer Vielzahl verschiedener Angebote
        für ein Smartphone soll die Preisspanne berechnet werden.
      \paragraph{Problemabstraktion} Gegeben ist eine Liste $p_1 \ldots p_n$ von
        Zahlen, die dem Preis der Angebote entsprechen. Gesucht ist die
        größtmögliche Differenz zwischen zwei Zahlen aus dieser Liste.
      \paragraph{Algorithmenentwurf} Die größte Differenz wird anhand der
      Differenz zwischen dem größten und kleinsten Element berechnet.\\
      \begin{lstlisting}[mathescape=true]
        setze $k = p_1$ //$k$: Kleinstes Element
        setze $g = p_1$ //$g$: Größtes Element
        setze $i = 1$
        solange $i \leq n$:
          wenn $p_i < k$, dann: $k = p_i$
          wenn $p_i > g$, dann: $g = p_i$
          erhöhe $i$ um $1$
        Ausgabe von $(g - k)$
      \end{lstlisting}
      \paragraph{Korrekrheitsnachweis, Verifikation} Da die Preisspanne die
        größtmögliche Differenz zwischen zwei Preisen ist, und der Algorithmus
        die größte und kleinste Zahl durch Iteration über alle Zahlen findet,
        ist dieser korrekt. Der Algorithmus terminiert, da er nur eine Schleife
        enthält, die genau $n$ mal wiederholt.
      \paragraph{Aufwandsanalyse} Die Schleife wird genau $n$ mal ausgeführt,
      und in jeder Ausführung werden 2 Vergleiche ausgeführt. Der Algorithmus
      benötigt daher $2n+3$ Schritte und steigt damit linear zur Anzahl der
      Angebote an.



\end{document}
