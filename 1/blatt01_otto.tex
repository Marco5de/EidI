\documentclass{article}

\usepackage{amsmath}
\usepackage{enumitem}
\usepackage[utf8]{inputenc}

\begin{document}
  \title{EidI Übungsblatt 1}
  \author{Jonas Otto}

  \maketitle

  \section{}
    \subsection{Aufgabe 1.1}
      \begin{itemize}
        \item $101010_{(2)}$\\
          Polynom:
          \begin{equation}
            1 \cdot 2^5 + 0 \cdot 2^4 + 1 \cdot 2^3 + 0 \cdot 2^2 + 1 \cdot 2^1 + 0 \cdot 2^0
          \end{equation}
          Horner Schema:
          \begin{equation}
            ((((1 \cdot 2 + 0) \cdot 2 + 1) \cdot 2 + 0) \cdot 2 + 1 ) \cdot 2 + 0 = 42
          \end{equation}
        \item $157_{(8)}$\\
          Polynom:
          \begin{equation}
            1 \cdot 8^2 + 5 \cdot 8^1 + 7 \cdot 8^0
          \end{equation}
          Horner Schema:
          \begin{equation}
            (1 \cdot 8 + 5) \cdot 8 + 7 = 111
          \end{equation}
        \item $ACDC_{(16)}$\\
          Polynom:
          \begin{equation}
            10 \cdot 16^3 + 12 \cdot 16^2 + 13 \cdot 16^1 + 12 \cdot 16^0
          \end{equation}
          Horner Schema:
          \begin{equation}
            \begin{aligned}
              ((A_{(16)} \cdot 16 + C_{(16)}) \cdot 16 + D_{(16)}) \cdot 16 + C_{(16)} &=\\
              ((10 \cdot 16 + 12) \cdot 16 + 13) \cdot 16 + 12 &= 44252
            \end{aligned}
          \end{equation}
        \item $10_{(10)}$\\
        Polynom:
        \begin{equation}
          1 \cdot 10^1 + 0 \cdot10^0
        \end{equation}
        Horner Schema:
          \begin{equation}
            \begin{aligned}
              (1 * 10) + 0 &=\\
              (1_{(2)} * 1010_{(2)}) + 0 &= 1010_{(2)}
            \end{aligned}
          \end{equation}
      \end{itemize}
    \subsection{Aufgabe 1.2}
      \paragraph{Problemspezifikation} Die Preisspanne bei verschiedenen Angeboten
      für ein Smartphone soll ermittelt werden.
      \paragraph{Problemabstraktion} Gegeben ist eine Liste $p_1 \ldots p_n$ von Zahlen. Gesucht
      ist die größtmögliche Differenz zwischen zwei Zahlen aus dieser Liste.
      \paragraph{Algorithmenentwurf} Die größte Differenz wird anhand der
      Differenz zwischen dem größten und kleinsten Element berechnet.\\

      $k = p_1$\\
      $g = p_1$\\
      $i = 1$\\
      solange $i \leq n$:\\
        \- wenn $p_i < k$: $k = p_i$\\
        \- wenn $p_i > g$: $g = p_i$\\
      Ausgabe von $(g - k)$


//TODO


\end{document}
