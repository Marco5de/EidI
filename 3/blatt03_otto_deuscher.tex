\documentclass{article}

\newcommand{\blattNr}{3}
\usepackage[utf8]{inputenc}
\usepackage[ngerman]{babel}
\usepackage{enumitem}
\usepackage{listings}
\usepackage{lstautogobble}
\usepackage{amsmath}
\usepackage{xcolor}

\definecolor{mygreen}{rgb}{0,0.6,0}
\definecolor{keyword}{HTML}{CB772F}
\definecolor{string}{HTML}{A5C25C}

\lstset{literate=
    {Ö}{{\"O}}1
    {Ä}{{\"A}}1
    {Ü}{{\"U}}1
    {ß}{{\ss}}1
    {ü}{{\"u}}1
    {ä}{{\"a}}1
    {ö}{{\"o}}1
    {~}{{\textasciitilde}}1,
    basicstyle=\small\ttfamily,
    numbers=left,
    breaklines=true,
    commentstyle=\color{mygreen},
    frame=single,
    keywordstyle=\color{keyword},
    stringstyle=\color{string}, 
}
\newcommand{\java}{\lstinline[language=Java]}

\setcounter{section}{\blattNr}

\newcommand{\uebungsblattTitel}{
    \title{EidI Übungsblatt \blattNr}
    \author{Jonas Otto, Marco Deuscher}
    \maketitle
}


\begin{document}
    \uebungsblattTitel

    \section*{}
        \subsection{}
            \begin{lstlisting}[autogobble]
                Setzte z = 1
                Während z <= 100
                    Setze t = (int) (z/2)
                    Während t > 1
                        Wenn z % t == 0, dann:
                            Abbruch der Schleife
                        t = t - 1
                    Wenn t == 1, dann:
                        Gib aus: z
                    z = z + 1
            \end{lstlisting}
        \subsection{}
            \begin{align}
                \begin{split}
                    3/5*-2&==3+5-2\\
                    3/5*(-2)&==(3+5)-2\\
                    (3/5)*(-2)&==(3+5)-2\\
                    ((3/5)*(-2))&==((3+5)-2)
                \end{split}
            \end{align}

            \begin{align}
                \begin{split}
                    \text{false}\ ||\ 8\ \%\ -3 &== 7 * 2\ \&\&\ \text{true}\\
                    \text{false}\ ||\ 8\ \%\ (-3) &== (7 * 2)\ \&\&\ \text{true}\\
                    \text{false}\ ||\ (8\ \%\ (-3)) &== (7 * 2)\ \&\&\ \text{true}\\
                    (\text{false}\ ||\ (8\ \%\ (-3))) &== ((7 * 2)\ \&\&\ \text{true})
                \end{split}
            \end{align}

        \subsection{}
            % TODO 3.3
            \begin{align}
                \begin{split}
                    System.out.println(0.3+0.3+0.3);
                        Ausgabe: 0.8999999999999999
                        Begründung: 
                    System.out.println('a'+'b'+'c'+"!");
                        Ausgabe: 294!
                        Begründung: 'a','b','c' sind char und werden aufgrund des + Operators als Integer behandelt
                                    entsprechend dem ASCII-Code ist a=97, b=98, c=99
                                    "!" ist ein String und wird auch als solcher ausgegeben
                    System.out.println(9/2);
                        Ausgabe: 4
                        Begründung: Es werden zwei Ganzzahlen durcheinander geteilt, folglich wird auch eine 
                                    Ganzzahl als Ergebnis ausgegeben. Die Nachkommastellen werden abgeschnitten.
                    System.out.println("Rechnung: " + 3 + -1 + 5);
                        Ausgabe: Rechnung: 3-15
                        Begründung: Da zuerst der String "Rechnung: " ausgegeben wird, wird alles folgende ebenfalls
                                    als String behandelt. Wollte man mit den Zahlen noch eine Rechenoperation ausführen
                                    müsste man entsprechend Klammern setzen.
                             
                \end{split}
            \end{align}

        \subsection{}
            \lstinputlisting[language=Java]{src/GrumpyCat/GrumpyCat.java}

\end{document}
